\documentclass{beamer}
\usetheme{Warsaw}
\usepackage{beamerthemesplit}
\usepackage{textcomp}
\title{Creating a Basic Document using \LaTeX}
\author{SPACE}
\date{\today}

\begin{document}
\maketitle

\begin{frame}
\frametitle{Basic Document}
%\begin{center}
\begin{block}{Try this...}
\textbackslash documentclass \{article\} \\
\textbackslash begin\{document\}\\
Small is beautiful.\\
\textbackslash end\{document\}
\end{block}
\begin{enumerate}
\item \textbackslash documentclass \{article\}
\item[•] This specifies what sort of document you intend to write.\{eg:article,letter,report etc\}
\item \textbackslash begin\{document\} 
\item[•] This begins the Document
\item \textbackslash end\{document\} 
\item[•] This ends the Document
\item The text in between will be the content of the Document
\end{enumerate}
\end{frame}

\begin{frame}
\frametitle{Compilation}
\begin{itemize}
\item Select pdflatex \{shortcut F6 \} to compile
\item To view pdf select Viewpdf or \{shortcut F7\}
\item we can also do this through commandline
\end{itemize}
\end{frame}

\begin{frame}{Line Breaking and Page Breaking}
\begin{enumerate}
\item To break a line in \LaTeX
\item[]\framebox{\textbackslash\textbackslash \hspace{0.2cm} or \hspace{0.2cm} \textbackslash newline}
\item To start a newpage
\item[] \framebox{\textbackslash newpage}
\end{enumerate}
\end{frame}

\begin{frame}{Special Characters \& Ready-Made Strings}
\begin{block}{Ready Made Strings}
\begin{enumerate}
\item[] \textbackslash today \hspace{0.2cm} - Current Date
\item[] \textbackslash TeX \hspace{0.2cm} - \TeX
\item[] \textbackslash LaTeX \hspace{0.2cm} - \LaTeX
\item[] \textbackslash LaTeXe \hspace{0.2cm} - \LaTeXe
\end{enumerate}
\end{block}
\end{frame}

\begin{frame}{Special Characters}
\begin{itemize}
\pause \item[] 
\begin{table}
\caption{Special Characters}
\centering
\begin{tabular}{c c}
\hline\hline
Special Character & \LaTeX Method \\[0.5 ex]
\hline
\# & \textbackslash \# \\
\$ & \textbackslash \$ \\
\& & \textbackslash \& \\
\% & \textbackslash \% \\
\^{} & \textbackslash \^{} \\
\_ & \textbackslash \_ \\
\{ & \textbackslash \{ \\
\} & \textbackslash \} \\[1 ex]
\hline
\end{tabular}
\end{table}
\vspace{0.2cm}
\pause \item How will you print \textbackslash 
\end{itemize}
\end{frame}

\begin{frame}{Quotes and Dashes}
\begin{block}{Quotation Marks}
\begin{itemize}
\pause \item Use two \`\ (grave accent)for opening quotation \\ mark and \'\ (vertical quote) for closing quotation mark.
\pause \item For single quotes just use one of each.
\pause \item try printing this line -- ``Please press the `x' key.''
\end{itemize}
\end{block}
\begin{block}{Dashes \& Hyphens}
\begin{itemize}
\pause \item Four kinds of Dashes
\begin{itemize}
\pause \item Hyphen $\rightarrow$ `-' \{a-b\}
\pause \item en-dash $\rightarrow$ `$--$' \{13 -- 27\}
\pause \item em-dash $\rightarrow$ `$---$' \{yes --- or no ?\}
\pause \item minus $\rightarrow$ `-' \{-1\}
\end{itemize}
\end{itemize}
\end{block}
\end{frame}

\begin{frame}{Some more special Characters..}
\begin{block}{How to print Tilde}
\begin{itemize}
\pause \item \~{} $\rightarrow$ \textbackslash\~{}\{\}
\end{itemize}
\end{block}
\begin{block}{How to print Degree Symbol}
\begin{itemize}
\pause \item $-30\,^{\circ}\mathrm{C}$  $\rightarrow$ \$-30\textbackslash,\^\ \{\textbackslash circ\}\textbackslash mathrm\{C\}\$
\pause \item try printing $50\,^{\circ}\mathrm{F}$
\pause \item Another easy way
\begin{itemize}
\pause \item \textbackslash usepackage\{textcomp\}
\pause \item 30 \textbackslash textcelsius\{\} is 86 \textbackslash textdegree\{\}F
\pause \item 30 \textcelsius{} is 86 \textdegree{}F
\end{itemize}
\end{itemize}
\end{block}
\end{frame}

\begin{frame}{Spacing \ldots}
\begin{block}{spacing}
\begin{itemize}
\pause \item A tilde ‘\~{}’ character generates a space that cannot be enlarged
\pause \item The additional space after periods can be disabled with the command  
\pause \item[] \framebox{\textbackslash frenchspacing}
\end{itemize}
\end{block}
\end{frame}
%------------------------------------------------------------------------------------------
\end{document}
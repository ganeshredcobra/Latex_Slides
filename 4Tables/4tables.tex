\documentclass{beamer}
\usetheme{Warsaw}
\usepackage{beamerthemesplit}
\usepackage{color}
\usepackage{textcomp}
\title{Tables in \LaTeX}
\author{SPACE}
\date{\today}
%\logo{\includegraphics[width=1cm]{space}}
\begin{document}
\maketitle
\begin{frame}{Tables}
\setbeamercovered{transparent}
Simple tables can be created in beamer with the tabular environment.
\begin{itemize}
\pause \item Tables start with the command \textbackslash begin\{tabular\}~\{ccc\}.
\pause \item \{ccc\}~tells us the number of columns as well as the alignment
of each column. This table has three columns, each column is
center aligned.
\pause \item Columns can be aligned to the left \{l\}, center \{c\}, or right \{r\}.
\pause \item Alignments can be mixed up. For example, \{lcrrr\}.
\pause \item Tables are constructed in rows. A \& divides each cell and each
row must end with \textbackslash \textbackslash.
\pause \item \textbackslash end\{tabular\}
\end{itemize}
\end{frame}

\begin{frame}{Table Example}
\setbeamercovered{transparent}
\begin{itemize}
\pause \item[]\begin{block}{Example}
\textbackslash begin\{tabular\}\{ccc\}\\
cell 1 ~\&~ cell 2~ \&~ cell 3 ~\textbackslash \textbackslash~~\\
cell 4 ~\&~ cell 5~ \&~ cell 6 ~\textbackslash \textbackslash~~\\
\textbackslash end\{tabular\}
\end{block}
\vspace{0.2cm}
\pause \item[]\begin{tabular}{ccc}
cell 1 & cell 2 & cell 3 \\
cell 4 & cell 5 & cell 6 \\
\end{tabular}
\end{itemize}
\end{frame}

\begin{frame}{hline Example}
\setbeamercovered{transparent}
\begin{itemize}
\item We can add \textbackslash hline between rows to divide rows more clearly:
\pause \item[]\begin{block}{Example}
\textbackslash begin\{tabular\}\{ccc\}\\
~~ \textcolor{red}{\textbackslash hline}\\
cell 1 ~\&~ cell 2~ \&~ cell 3 ~\textbackslash \textbackslash~~\\
~~ \textcolor{red}{\textbackslash hline}\\
cell 4 ~\&~ cell 5~ \&~ cell 6 ~\textbackslash \textbackslash~~\\
~~ \textcolor{red}{\textbackslash hline}\\
\textbackslash end\{tabular\}
\end{block}
\vspace{0.2cm}
\pause \item[]\begin{tabular}{ccc}
\hline
cell 1 & cell 2 & cell 3 \\
\hline
cell 4 & cell 5 & cell 6 \\
\hline
\end{tabular}
\end{itemize}
\end{frame}

\begin{frame}{Pipe Example}
\setbeamercovered{transparent}
\begin{itemize}
\item We can add a ``$|$'' between column indicators to divide columns more
clearly:
\pause \item[]\begin{block}{Example}
\textbackslash begin\{tabular\}\{$|c|c|c|$\}\\
cell 1 ~\&~ cell 2~ \&~ cell 3 ~\textbackslash \textbackslash~~\\
cell 4 ~\&~ cell 5~ \&~ cell 6 ~\textbackslash \textbackslash~~\\
\textbackslash end\{tabular\}
\end{block}
\vspace{0.2cm}
\pause \item[]\begin{tabular}{|c|c|c|}
cell 1 & cell 2 & cell 3 \\
cell 4 & cell 5 & cell 6 \\
\end{tabular}
\end{itemize}
\end{frame}

\begin{frame}{Column Spacing}
\setbeamercovered{transparent}
\begin{itemize}
\pause \item Column width can be modified by changing \textbackslash tabcolsep like this
\pause\item[]\begin{columns}[c]
\column{1.5in}
\textbackslash begin\{tabular\}\{$|l|l|$\}\\
~~\textbackslash hline \\
~~a \& Row 1 \textbackslash\textbackslash ~~\textbackslash hline \\
~~b \& Row 2 \textbackslash\textbackslash ~~\textbackslash hline \\
~~c \& Row 3 \textbackslash\textbackslash ~~\textbackslash hline \\
\textbackslash end\{tabular\}\\
\vspace{0.3cm} \underline{Output :}\\
\begin{tabular}{|l|l|}
  \hline
  a & Row 1 \\ \hline
  b & Row 2 \\ \hline
  c & Row 3 \\ \hline
\end{tabular}
\column{2in}
\textbackslash setlength \{\textbackslash tabcolsep\}\{10pt\}\\
%\textbackslash setlength\{\textbackslash extrarowheight\}\{1.5pt\}
\textbackslash begin\{tabular\}\{$|l|l|$\}\\
~~\textbackslash hline \\
~~a \& Row 1 \textbackslash\textbackslash ~~\textbackslash hline \\
~~b \& Row 2 \textbackslash\textbackslash ~~\textbackslash hline \\
~~c \& Row 3 \textbackslash\textbackslash ~~\textbackslash hline \\
\textbackslash end\{tabular\}\\
\vspace{0.3cm} \underline{Output :}\\
\setlength{\tabcolsep}{10pt}
%\setlength{\extrarowheight}{1.5pt}
\begin{tabular}{|l|l|}
  \hline
  a & Row 1 \\ \hline
  b & Row 2 \\ \hline
  c & Row 3 \\ \hline
\end{tabular}
\end{columns}
\end{itemize}
\end{frame}

\begin{frame}{Rotating a Table \ldots}
\setbeamercovered{transparent}
\begin{itemize}
\pause \item \textbackslash usepackage ~\{rotating\}
\pause \item[] \begin{block}{sidewaystable}
\textbackslash begin~\{sidewaystable\}\\
\textbackslash begin\{tabular\} \\
\ldots \\
\textbackslash end \{tabular\}\\
\textbackslash end~\{sidewaystable\}
\end{block}
\pause \item[] \begin{block}{rotate}
\textbackslash begin \{rotate\}~\{angle\}\\
\textbackslash begin\{tabular\} \\
\ldots \\
\textbackslash end \{tabular\}\\
\textbackslash end \{rotate\}
\end{block}
\end{itemize}
\end{frame}

\begin{frame}{Rotation Continued \ldots}
\setbeamercovered{transparent}
\begin{itemize}
\pause \item \textbackslash usepackage \{pdflscape\}
\pause \item[] \begin{block}{pdflscape}
\textbackslash begin \{landscape\}\\
......\\
\textbackslash end \{landscape\}
\end{block}
\pause \item This makes the page appear as “landscape”, while the rest will remain in “portrait” orientation.
\end{itemize}
\end{frame}
\begin{frame}{Exporting a Table from OOo Spreadsheet}
\setbeamercovered{transparent}
\begin{itemize}
\pause \item Visit the site \textcolor{red}{http://calc2latex.sourceforge.net/}
\pause \item Download the package and follow the installation instructions
\pause \item Make your desired Table in Calc
\pause \item Select cells you want to convert into LaTeX format.
\pause \item Goto Tools~\textgreater~Macros~\textgreater~Run Macro
\pause \item Inside MyMacros library name select \framebox{calc2latex} and in macro name select \framebox{main} and click run
\pause \item A settings tab will appear click convert you will get the \LaTeX code
\end{itemize}
\end{frame}

%\begin{frame}{Long Table \ldots}
%\setbeamercovered{transparent}
%\begin{itemize}
%\pause \item The longtable package is designed to make tables that span page breaks.
%\end{itemize}
%\end{frame}

%\begin{frame}{Footnotes in Tables}
%\setbeamercovered{transparent}
%\begin{itemize}
%\item content...
%\end{itemize}
%\end{frame}

\end{document}
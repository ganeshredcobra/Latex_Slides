\documentclass{beamer}
\usetheme{Warsaw}
\usepackage{beamerthemesplit}
\usepackage{color}
\usepackage{colortbl}
\usepackage{textcomp}
\usepackage{graphicx}
\usepackage{subfigure}
\title{Color Table's \\ Using \LaTeX}
\author{SPACE}
\date{\today}
%\logo{\includegraphics[width=1cm]{space}}
\begin{document}
\maketitle

\begin{frame}{Color Tables }
\begin{block}{package}
\begin{itemize}
\item \textbackslash usepackage\{colortbl\}
\item The colortbl package provides a number of commands using which one can obtain really
colorful tables.
\end{itemize}
\end{block}
\end{frame}

\begin{frame}{color table's}
\setbeamercovered{transparent}
\begin{itemize}
\pause\item[]\begin{columns}[c]
\column{1.5in}
\textbackslash begin\{tabular\}\{$|l|r|$\}\\
~~\textbackslash hline \\
~~one \& two  \textbackslash\textbackslash ~~\textbackslash hline \\
~~three \& four\textbackslash\textbackslash ~~\textbackslash hline \\
\textbackslash end\{tabular\}\\
\vspace{0.3cm} \underline{Output :}\\
\begin{tabular}{|l|l|}
  \hline
  one & two \\ \hline
  three & four \\ \hline
\end{tabular}
\column{1.5in}
%\textbackslash setlength\{\textbackslash extrarowheight\}\{1.5pt\}
\textbackslash begin\{tabular\}\{$|l|r|$\}\\
~~\textbackslash hline \\
~~\textbackslash rowcolor\{blue\} one \& two  \textbackslash\textbackslash ~~\textbackslash hline \\
~~three \& four\textbackslash\textbackslash ~~\textbackslash hline \\
\textbackslash end\{tabular\}\\
\vspace{0.3cm} \underline{Output :}\\
%\setlength{\tabcolsep}{10pt}
%\setlength{\extrarowheight}{1.5pt}
\begin{tabular}{|l|l|}
  \hline
\rowcolor{blue} one & two \\ \hline
\rowcolor{green} three & four \\ \hline
\end{tabular}
\end{columns}
\end{itemize}
\end{frame}

\begin{frame}{color table's}
\setbeamercovered{transparent}
\begin{itemize}
\item \textbackslash rowcolor\{color\}
\item \textbackslash cellcolor\{color\}
\item[] Examples :
\item[] \begin{tabular}{|l|l|}
  \hline
\rowcolor{yellow} one & two \\ \hline
\rowcolor{red} three & four \\ \hline
\end{tabular} $\rightarrow$ Rowcolor
\item[] \begin{tabular}{|l|l|}
  \hline
\cellcolor{green} {one} & two \\ \hline
 three & \cellcolor{blue} \color{yellow} {four} \\ \hline
\end{tabular} $\rightarrow$ Cell color *Text color
\end{itemize}
\end{frame}

\begin{frame}{Try this....}
\begin{tabular}{|l|r|r|r|r|}
\hline
\rowcolor{green} Name & Class & Mark1 & Mark2 & Mark3\\\hline
Albert & 8 & 40 & 35 & 30 \\\hline
Edwin & 8 & \cellcolor{yellow} 27 & 42 & 39 \\\hline
Peter & 8 & 41 & 39 & 40 \\\hline
Harry & 8 & 31 & 37 & \cellcolor{red}15 \\\hline
\end{tabular}
\end{frame}

\end{document}